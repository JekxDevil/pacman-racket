\documentclass{article}
\usepackage{graphicx}

\title{
\textbf{pacman-racket} \\
A pragmatic implementation of Pacman in Racket
}

\author{
    Alessandro Zanzi,
    Filippo Piloni,\\
    Jeferson Morales Mariciano,
    Paolo Deidda
}
\date{
USI \\
Faculty of Informatics \\
[\baselineskip]  2021/2022
}


\begin{document}
%%%%%%%%%%%%%%%%%%%%%%%
%%%%%%%%%%%%%%%%%%%%%%%
%%%%%%%%%%%%%%%%%%%%%%%
\begin{titlepage}
\maketitle  

\end{titlepage}
 %%%%%%%%%%%%%%%%%%%%%%%
 %%%%%%%%%%%%%%%%%%%%%%%
 %%%%%%%%%%%%%%%%%%%%%%%
  \begin{abstract}
Conceptualization, research and thinking out-of-the-box
to implement the well-known classic game Pac-man
using the Racket functional programming language.\\
To develop this project, our group used abstractions and open-source tools to develop five files using Racket and linking them together to clarify the divisions between the various parts of the program and their respective functions.\\
To better organize with this project, we used several instruments such as GitHub to be able to work at the program at any momen t remaining updated.\\
Using the knowledge developed during the lectures and the guides provided by Racket Documentation, the result of the project was a working revisited version of the game Pac-Man composed by one level where the ghosts move using a contrast logic that considers Pac-Man position in the map.

 \end{abstract}
 \clearpage
  \tableofcontents
 \clearpage

\section{Guide}

\subsection{Game logic}
 The player must guide a yellow spherical creature, called Pac-Man, making it eat all the numerous dots scattered inside the maze and, in doing so, he must avoid being touched by four ghosts, or the game will be over. In order to make the game easier for the player, there are four "power pellets" in the corners of the screen that turn the situation by making Pac-Man able to the ghosts without ending the game. Each dot taken will increase the score of 10 points instead the cherries and powerpellets are worth 100 points.

\subsection{Key guide}
The player interact with Pac-Man with the four keys \texttt{up}, \texttt{down}, \texttt{left} and \texttt{right}, to move in the respective direction.
\\
Is possible to press the key \texttt{Q} to end the game at any time.

 
  \section{Usage}
  
  \subsection{Setting}
To run the game the player need a DrRacket IDE.\\
\\
To make the process easier the user can create an executable file to have a clickable file which will start the game automatically when ckicked. To create this file there are two ways.\\
\\
The first require the drRacket compiler, once open upload the \textit{main.rkt}, click on \textit{Racket} above in the toolbar and finally click \textit{Create Executable...} in the drop-down menu.\\
\\
For the second way is sufficient to open the terminal, go to the \textit{pac-man} folder and input the following code:\\
\\
\texttt{raco exe main.rkt}

\subsection{Start and Stop}
In order to start the game you have to open the \textit{main.rkt} file and press the \textit{run} button or press together the two keys \textit{ctrl} and \textit{R}.\\
If the executable file has been installed, to run the game will be sufficient to double click on it. 
\\
When a win or a loss occur the player quit the game with by closing the tab, to play another round the player can run the program again as indicated above.

 \end{document}
